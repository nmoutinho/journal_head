\subsection{Related Work}

% -- This section needs work. A journal, as compared to a conference, requires comprehensive well developed related work.

% Start by the "body schema". Cite a seminal work, in order to show it is a well known, classical, problem / objective, which will be widespread due to mass consumer market

% current line of refs: Hersch08, Cantin10, Tworek08, Santos10

Seminal works addressed the robot self-calibration problem as non-linear parameter estimation problems given sufficient input data from the robot sensory system. For instance, the so-called Body Schema, a denomination for the set of kinematic parameters that form the robot model, including link lengths and angles in addition to joints offsets, have been estimated with local optimization methods given appropriate initializations. 

In \cite{Hersch08} the authors present an online learning system for the body schema of the Hoap3 robot, a humanoid robotic platform with 24 degrees of freedom (DOF). The system uses information from the propriosensors and from stereo vision to correct and calibrate its internal model, by tracking its end-effectors using color markers. Although the authors show good results in simulation, for the considered approach, they recognize it is difficult to implement such a solution in a real robot with many DOF since it would take a large amount of time for the robot to explore all its joints space, by randomly moving its end-effectors. In particular, for some of its end-effectors the acquired information may be insufficient due to lack of direct visibility (the robot may not be able to always observe its feet). The random exploration of the whole joints space is widely time-consuming and may not provide enough information for the system to correctly converge. An intelligent exploration of this space could highly improve the calibration results while reducing the total calibration time. 

The authors in \cite{Cantin10} present an online active learning algorithm for the body schema of a robot that uses markers in its end-effectors. However, the main contribution of that work resides in the way the robot explores its joints space. This exploration, contrary to other approaches using active learning techniques, is not random but selective in the acquired observation. The robot will only get observations that could actually improve its calibration and explore areas where uncertainty remains high. The accuracy and speed improvement by using this technique is considerable when comparing with a passive approach where the robot randomly explored its joints space. It is of utmost importance to understand if an observation will actually add useful information to the system or simply act as noise. 

In the work described in \cite{Tworek08} the authors present a kinematic calibration system for a pan-tilt structure with a camera as its end-effector. The system estimates the homographies between consecutive time-instants by tracking points on the images. This method allows the calibration of the joint angles offsets that were present at startup due to relative encoders. Robotic platforms with relative encoders require a kinematic calibration before any operation and in these cases, calibration time is crucial. The authors in \cite{Santos10} present a kinematic calibration algorithm for a robotic head, similar to the one used in our thesis. This robotic head is equipped with relative encoders and they propose a solution to estimate the offsets of each joint, from the neck base to each of the eyes. The problem is separated into two sub-problems that are solved using different methods, more suitable for each case. To calibrate the head the authors use information from the inertial sensor in order to find the joints offsets that enables the alignment of the head with the robot's body. To calibrate the eyes the authors used the same approach as in \cite{Tworek08}. None of the implemented methods take noise into consideration which may lead to erroneous estimations. Moreover, these approaches were not designed to be used in an online manner and can not respond or adapt to changes that may occur during operation. Due to heavy use these platforms usually have slow mechanical drifts in their motor shafts due to wear, impact and strain that deform the parts, where the encoders can not provide any measurements. An online calibration system can detect these changes and adapt its estimates along time as we will show in this thesis.

From the literature we can see that most solutions require special markers to calibrate the robotic platforms or simply cannot calibrate it in an online fashion. In this work we present a full calibration system for a robotic head platform that aims to calibrate its kinematic model in an online fashion, without any special markers, using only natural information from the environment and measurements from its embedded sensors. Our system can be left running during the whole operation of the robot, in order to adapt to any changes that may occur.

